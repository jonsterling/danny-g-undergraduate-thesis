\section{An ML-like Language with General References}\label{sec:language}

In order to make the discussion of logical relations more concrete, a
particular language is necessary. In this section we develop a core
calculus suitable for studying the effects of general references on
reasoning.

The language under consideration here is heavily influenced by the
Modernized Algol discussed by Harper~\citep{TODO:PFPL}. It features a
syntactic separation between commands and expressions. Expressions are
characterized by being unable to depend on in any way on the
heap. Commands may modify the heap using assignables, a mutable
``variable'' that makes clear the binding structure (nominal rather
than substitutive). A crucial component of the system is a modality
for internalizing and suspending commands to treat them as
expressions. Arguably this language is heavier-weight than the ML-like
languages usually discussed which just allow references in any
location. It does confer that advantage that handling state is now a
question of defining the logical action of the command modality and
its complexity is not littered throughout the full logical
relation. Indeed, the definitions of the logical relation at $\to$ or
$\forall$ are almost completely unchanged.

The syntax of our language has three sorts: commands, expressions, and
types defined by the following grammar.
\[
  \begin{array}{lcl}
    \tau & ::= & \alpha \mid \fn{\tau}{\tau} \mid \allNoKind{\alpha}{\tau}
    \mid \cmd{\tau}\\
    c & ::= & \ret{e} \mid \get{\alpha} \mid \set{\alpha}{e} \\
      & \mid & \dcl{\alpha}{e}{c} \mid \bnd{x}{e}{c}\\
    e & ::= & x \mid \ap{e}{e} \mid \lam{x}{\tau}{e} \mid
              \LamNoKind{\alpha}{e} \mid \cmd{e}
  \end{array}
\]
Only one point of this syntax must be clarified, which is the
unfortunate coincidence of $\alpha$s. In an attempt to maintain
consistency with the standard literature on System F and
Harper~\citep{TODO:PFPL} $\alpha$ here refers either to an assignable
(a symbol) or a type variable. Crucially, assignables are \emph{not}
variables. Assignables are bound by the operators of our language and
they do $\alpha$-vary as a variable might but they may not be
substituted for. This justifies comparing bound symbols for equality;
they are placeholders and contain an intrinsic notion of identity in
the form of a binding site. It is nonsensical to talk about a rule
like the following.
\[
  \inferrule{ }{\step{\dcl{\alpha}{v}{c}}{[v/\alpha]c}}
\]
In some sense, this is similar to the confusion that many programmers
have when discussing languages with variables: C-like languages do not
permit substitution because they do not possess variables but rather
assignables. In C a rule like the above is clearly false and leads to
statements such as \verb+1 = 2+. One can understand the difference
between $\bnd{x}{e}{c}$ and $\dcl{x}{e}{c}$ as the difference between
\emph{variables}, defined through substitution, and \emph{assignables}
which are defined through binding identity. The former is like a
let-binding while the latter is closer to a declaration in C. In our
language this separation exists which is why the operators for reading
and writing a mutable cell are indexed by symbol rather than taking an
arbitrary term. There is much to be said on the subject of symbols
and, more generally, nominal binding but it is sadly out-of-scope for
this discussion. The interested reader is referred
to~\citep{TODO-NOMINAL-STUFF}.

The static semantics of the language are divided into three judgments,
the first of which is the judgment ensuring a type is a
well-formed. Informally, a type is well formed in a context $\Delta$
if $\Delta$ contains all the free variables of the type.
\begin{mathpar}
  \declareJudgement{\hasTJ{}{\Delta}{\tau}{\tp}}\\
  \inferrule{
    \alpha \in \Delta
  }{\hasTJ{}{\Delta}{\alpha}{\tp}}\and
  \inferrule{
    \hasTJ{}{\Delta}{\tau_1}{\tp}\\
    \hasTJ{}{\Delta}{\tau_2}{\tp}
  }{\hasTJ{}{\Delta}{\fn{\tau_1}{\tau_2}}{\tp}}\and
  \inferrule{
    \hasTJ{}{\Delta, \alpha}{\tau}{\tp}
  }{\hasTJ{}{\Delta}{\allNoKind{\alpha}{\tau}}{\tp}}\and
  \inferrule{
    \hasTJ{}{\Delta}{\tau}{\tp}
  }{\hasTJ{}{\Delta}{\cmd{\tau}}{\tp}}
\end{mathpar}
In order to explain the statics of expressions and commands two
judgments are necessary and they must depend on each other. This
mutual dependence is a result of the $\cmd{-}$ modality which
internalizes the command judgment. First the expression judgment is
given, it is completely standard except that it must also be fibered
over a specification of the available assignables. This extra context
is necessary in order to make sense of the binding done in
$\mathsf{dcl}$.
\begin{mathpar}
  \declareJudgement{\hasESigJ{\Delta}{\Gamma}{\Sigma}{e}{\tau}}\\
  \inferrule{
    x : \tau \in \Gamma
  }{\hasESigJ{\Delta}{\Gamma}{\Sigma}{x}{\tau}}\and
  \inferrule{
    \hasESigJ{\Delta}{\Gamma, x : \tau_1}{\Sigma}{e}{\tau_2}
  }{\hasESigJ{\Delta}{\Gamma}{\Sigma}{\lam{x}{\tau_1}{e}}{\fn{\tau_1}{\tau_2}}}\and
  \inferrule{
    \hasESigJ{\Delta}{\Gamma}{\Sigma}{e_1}{\fn{\tau_1}{\tau_2}}\\
    \hasESigJ{\Delta}{\Gamma}{\Sigma}{e_2}{\tau_1}\\
  }{\hasESigJ{\Delta}{\Gamma}{\Sigma}{\ap{e_1}{e_2}}{\tau_2}}\and
  \inferrule{
    \hasESigJ{\Delta, \alpha}{\Gamma}{\Sigma}{e}{\tau}
  }{\hasESigJ{\Delta}{\Gamma}{\Sigma}{\LamNoKind{\alpha}{e}}{\allNoKind{\alpha}{\tau}}}\and
  \inferrule{
    \hasESigJ{\Delta}{\Gamma}{\Sigma}{e}{\allNoKind{\alpha}{\tau_1}}\\
    \hasTJ{}{\Delta}{\tau_2}{\tp}
  }{\hasESigJ{\Delta}{\Gamma}{\Sigma}{\Ap{e}{\tau_2}}{[\tau_2/\alpha]\tau_1}}\and
  \inferrule{
    \hasMJ{\Delta}{\Gamma}{\Sigma}{m}{\tau}
  }{\hasESigJ{\Delta}{\Gamma}{\Sigma}{e}{\tau}}
\end{mathpar}
These are the rules of System F with the exception of the final
one. This makes use of the judgment for commands, where
$\hasM{m}{\tau}$ signifies that $m$ is a command which (when executed
on the appropriate heap) will run to a $\ret{v}$ for some
$\hasE{v}{\tau}$. It is worth noting that $\Sigma$ in this judgment is
unimportant except in this rule as well. It is only necessary to track
the available assignables because without this information it is not
possible to determine if a command is well-typed. The rules for
commands are as follows.
\begin{mathpar}
  \inferrule{
    \hasESigJ{\Delta}{\Gamma}{\Sigma}{e}{\tau}
  }{\hasMJ{\Delta}{\Gamma}{\Sigma}{\ret{e}}{\tau}}\and
  \inferrule{
    \alpha \div \tau \in \Sigma
  }{\hasMJ{\Delta}{\Gamma}{\Sigma}{\get{\alpha}}{\tau}}\and
  \inferrule{
    \alpha \div \tau \in \Sigma\\
    \hasESigJ{\Delta}{\Gamma}{\Sigma}{e}{\tau}
  }{\hasMJ{\Delta}{\Gamma}{\Sigma}{\set{\alpha}{e}}{\tau}}\and
  \inferrule{
    \hasESigJ{\Delta}{\Gamma}{\Sigma}{e}{\tau_2}\\
    \hasMJ{\Delta}{\Gamma}{\Sigma, \alpha : \tau_2}{m}{\tau_1}
  }{\hasMJ{\Delta}{\Gamma}{\Sigma}{\dcl{\alpha}{e}{m}}{\tau_1}}\and
  \inferrule{
    \hasESigJ{\Delta}{\Gamma}{\Sigma}{e}{\cmd{\tau_1}}\\
    \hasMJ{\Delta}{\Gamma, x : \tau_1}{\Sigma}{m}{\tau_2}
  }{\hasMJ{\Delta}{\Gamma}{\Sigma}{\bnd{x}{e}{m}}{\tau_2}}\and
\end{mathpar}
The operational semantics of this language can be presented in two
distinct ways differing in the way that the allocation of a fresh heap
cell is done. In classical literature, an expression is evaluated
using traditional small step
semantics~\citep{TODO-SMALL-STEP}. Commands in this framework are
represented as a pair $(m, h)$, a command and the heap it is running
on, and there is another stepping relation defined between these
configurations. In \citet{TODO-PFPL} the binding structure of symbols is
preserved in the operational semantics so that a running command is
represented as a configuration $\nu \Sigma.\ (m, h)$ where $\nu$ is a
collection of symbols bound in both $m$ and $h$. The advantage of this
presentation is that it eliminates the choice of how to pick a fresh
symbol during allocation. In the more traditional set up the stepping
relation nondeterministic, allowing for any fresh location to be
chosen. This nondeterminism complicates many metatheoretic properties
and obscures what is really happening since the language is
fundamentally not nondeterministic! The formal presentation has just
obscured the precise meaning of a construct and approximated it with
nondeterminism. For the problems discussed in this thesis, however,
this advantage is not significant and it proves to be (while not
impossible) onerous to convert from semantic objects to $\Sigma$ when
running programs. Indeed a slightly different presentation from Harper
would be required in any case since programs will be run on heaps
containing locations varying over not syntactic but semantic
types. This means for expository purposes the traditional operational
semantics are chosen.
\begin{mathpar}
  \declareJudgement{\valueJ{v}}\\
  \inferrule{ }{\valueJ{\lam{x}{\tau}{e}}}\and
  \inferrule{ }{\valueJ{\LamNoKind{\alpha}{e}}}\and
  \inferrule{ }{\valueJ{\cmd{m}}}\\
  \declareJudgement{\step{e}{e'}}\\
  \inferrule{
    \step{e_1}{e_1'}
  }{\step{\ap{e_1}{e_2}}{\ap{e_1'}{e_2}}}\and
  \inferrule{
    \valueJ{v}\\
    \step{e_2}{e_2'}
  }{\step{\ap{v}{e_2}}{\ap{v}{e_2'}}}\and
  \inferrule{
    \valueJ{v}
  }{\step{\ap{(\lam{x}{\tau}{e})}{v}}{[v/x]e}}\and
  \inferrule{
    \steps{e}{e'}
  }{\steps{\Ap{e}{c}}{\Ap{e'}{c}}}\and
  \inferrule{ }{\step{\ap{(\LamNoKind{\alpha}{e})}{c}}{[c/\alpha]e}}\\
  \declareJudgement{\finalJ{m}{h}}\\
  \inferrule{
    \valueJ{v}
  }{\finalJ{m}{h}}\\
  \declareJudgement{\stepsM{m}{h}{m'}{h'}}\\
  \inferrule{
    \step{e}{e'}
  }{\stepM{\ret{e}}{h}{\ret{e'}}{h}}\and
  \inferrule{
    \step{e}{e'}
  }{\stepM{\bnd{x}{e}{m}}{h}{\bnd{x}{e'}{m}}{h}}\and
  \inferrule{
    \stepM{m_1}{h}{m_1'}{h'}
  }{\stepM{\bnd{x}{\cmd{m_1}}{m_2}}{h}{\bnd{x}{\cmd{m_1'}}{m_2}}{h'}}\and
  \inferrule{
    \valueJ{v}
  }{\stepM{\bnd{x}{\cmd{\ret{v}}}{m}}{h}{[v/x]m}{h}}\and
  \inferrule{
    \step{e}{e'}
  }{\stepM{\dcl{\alpha}{e}{m}}{h}{\dcl{\alpha}{e'}{m}}{h}}\and
  \inferrule{
    \valueJ{v}\\
    \alpha \disjoint h
  }{\stepM{\dcl{\alpha}{v}{m}}{h}{m}{h[\alpha \mapsto v]}}\and
  \inferrule{ }{\stepsM{\get{\alpha}}{h}{\ret{h(\alpha)}}{h}}\and
  \inferrule{
    \step{e}{e'}
  }{\stepsM{\set{\alpha}{e}}{h}{\set{\alpha}{e'}}{h}}\and
  \inferrule{
    \valueJ{v}
  }{\stepsM{\set{\alpha}{v}}{h}{\ret{v}}{h[\alpha \mapsto v]}}\and
\end{mathpar}
This gives a precise account of the language which we want to
construct a logical relation for. Still undefined, however, is the
notion of equivalence that this logical relation should capture. It
was informally sketched in the introduction as the ability to replace
one program with the other in an arbitrary context. It is now possible
to make this definition formal. First, a context formally is a term
with a distinguished hole in it, written $\hole$. They are generated
from the following grammar.
\[
  \begin{array}{lcl}
    C &::=& \hole \mid \ap{C}{e} \mid \ap{e}{C} \mid \lam{x}{\tau}{C} \mid
            \LamNoKind{\alpha}{C} \mid \Ap{C}{c} \mid \cmd{C_m}\\
    C_m &::=& \ret{C} \mid \bnd{x}{C}{m} \mid \bnd{x}{e}{C_m}\\
      & \mid & \dcl{\alpha}{C}{m} \mid \dcl{\alpha}{e}{C_m} \mid \set{\alpha}{C}\\
    C^m &::=& \ap{C^m}{e} \mid \ap{e}{C^m} \mid \lam{x}{\tau}{C^m} \mid
    \LamNoKind{\alpha}{C^m} \mid \Ap{C^m}{c} \mid \cmd{C_m^m}\\
    C_m^m &::=& \hole \mid \ret{C^m} \mid \bnd{x}{C^m}{m} \mid \bnd{x}{e}{C_m^m}\\
 & \mid & \dcl{\alpha}{C^m}{m} \mid \dcl{\alpha}{e}{C_m^m} \mid \set{\alpha}{C^m}\\
  \end{array}
\]
Two of varieties contexts, $C$ and $C_m$, come equipped with a natural
operation replacing $\hole$ with an expression, denoted $C[e]$, while
the other two possess an identical operation for commands. Notice that
crucially this operation is capturing; $e$ is allowed to make use of
the variables bound in $C$. This is crucial to allowing contextual
equivalence to work with open terms. A small point of interest is that
the fact that we have two sorts which may mention each other leads to
four different varieties of contexts. This quadratic growth continues
which makes contextual equivalence difficult to define for complex
languages. This has been addressed in Crary~\citep{TODO-MODULES} by
defining contextual equivalence as the unique relation satisfying
Theorem~\ref{thm:language:cxt}.
n
Finally, contexts can be equipped with a notion of typing. This typing
relation for $C$ for instance expresses that when filled with
a program $\hasESigJ{\Delta}{\Gamma}{\Sigma}{e}{\tau}$ they produce a
term $\hasESigJ{\Delta'}{\Gamma'}{\Sigma'}{C[e]}{\tau'}$. This typing
relation is spelled out below.
\begin{mathpar}
  todo
\end{mathpar}
With this, we are in a position to define contextual equivalence for
our language.
\begin{defn}\label{def:language:heapmatch}
  A heap, $h$, is said to match $\Sigma$, written $h : \Sigma$, if for
  all $\alpha \div \tau \in \Sigma$,
  $\hasESigJ{\cdot}{\cdot}{\Sigma}{h(l)}{\tau}$.
\end{defn}
\begin{defn}\label{def:language:kleene}
  To programs are said to be Kleene equal at $\Sigma$, written
  $e_1 \simeq_\Sigma e_2$ if for all $h : \Sigma$ and $m_1, m_2$ so
  that $\steps{e_i}{m_i}$ then $(m_1, h) \Downarrow$ and
  $(m_2, h) \Downarrow$ or $(m_1, h) \Uparrow$ and
  $(m_2, h) \Uparrow$.
\end{defn}
This definition of Kleene equality is slightly non-standard in that it
requires that the expressions are commands and that it only checks if
the coterminate, not that they terminate at the same value. This
deviation is due to the fact that there is no base observable type in
this language and so there is no observation to by made besides
termination. Furthermore, it is not hard to prove that expressions
always terminate (they behave as System F with a new base type) so
\emph{no} interesting observations may be made of just expressions. We
will also make use of the notation $(m_1, h_1) \simeq (m_2, h_2)$ to
signify just the final part of this definition. Finally, we can lift
Kleene equality to a definition of contextual equivalence.
\begin{defn}\label{def:language:cxt}
  To programs are contextually equivalent,
  $\equivESigJ{\Delta}{\Gamma}{\Sigma}{e_1}{e_2}{\tau}$ if for all
  contexts
  $\hasCEJ{\Delta}{\Gamma}{\Sigma}{\tau}{C}{\cdot}{\cdot}{\Sigma'}{\cmd{\tau'}}$,
  $C[e_1] \simeq_{\Sigma'} C[e_2]$.

  Similarly, two commands are said to contextually equivalent if
  $\hasCEJ{\Delta}{\Gamma}{\Sigma}{\tau}{C^m}{\cdot}{\cdot}{\Sigma'}{\cmd{\tau'}}$,
  $C^m[m_1] \simeq_{\Sigma'} C^m[m_2]$.
\end{defn}
This definition formalizes our earlier intuitions and it makes obvious
the issues that were mentioned with contextual equivalence previously.
In order to establish contextual equivalence we must write a proof
which handles an arbitrary choice of $C$, clearly a difficult
task.

Most proofs of contextual equivalence make use of a coinductive
characterization of it. Contextual equivalence is the coarsest
consistent congruence relation on terms. A consistent relation is one
where if two terms are related by the relation than they
coterminate. A congruence relation is an equivalence relation
(reflexive, symmetric, transitive) which respects the structure of the
terms. To make this precise, a pair of relations $(\relR, \relS)$ is
said to respect the structure of terms if the following inferences are
valid.
\begin{mathpar}
  \inferrule{
    e_1 \relR e_2
  }{\lam{x}{\tau}{e_1} \relR \lam{x}{\tau}{e_2}}\and
  \inferrule{
    e_1 \relR e_1'\\
    e_2 \relR e_2'
  }{\ap{e_1}{e_2} \relR \ap{e_1'}{e_2'}}\and
  \inferrule{
    e_1 \relR e_2
  }{\LamNoKind{\alpha}{e_1} \relR \LamNoKind{\alpha}{e_2}}\and
  \inferrule{
    e_1 \relR e_2
  }{\Ap{e_1}{c} \relR \Ap{e_2}{c}}\and
  \inferrule{
    m_1 \relS m_2
  }{\cmd{m_1} \relR \cmd{m_2}}\and
  \inferrule{
    e_1 \relR e_2
  }{\ret{e_1} \relS \ret{e_2}}\and
  \inferrule{
    e_1 \relR e_2\\
    m_1 \relS m_2
  }{\bnd{x}{e_1}{m_1} \relS \bnd{x}{e_2}{m_2}}\and
  \inferrule{
    e_1 \relR e_2\\
    m_1 \relS m_2
  }{\dcl{\alpha}{e_1}{m_1} \relS \dcl{\alpha}{e_2}{m_2}}\and
  \inferrule{
    e_1 \relR e_2\\
  }{\set{\alpha}{e_1} \relS \set{\alpha}{e_2}}\and
  \inferrule{
  }{\get{\alpha} \relS \get{\alpha}}
\end{mathpar}
The standard result is then captured by the following theorem.
\begin{thm}\label{thm:language:cxt}
  Contextual equivalence on expressions and commands is the coarsest
  consistent congruence.
\end{thm}
This concludes the discussion of the language under consideration. We
now turn to discussing logical relations for this language.

%%% Local Variables:
%%% mode: latex
%%% TeX-master: "../main"
%%% End:
