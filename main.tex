\documentclass{amsart}
\usepackage{natbib}
\usepackage{amsthm}
\usepackage{amsmath}
\usepackage{amssymb}
\usepackage{mathpartir}
\usepackage{stmaryrd}
\usepackage{xifthen}
\usepackage{tikz-cd}
\usepackage{xcolor}

% Formatting tools
\newcommand{\declareJudgement}[1]{\framebox{$\displaystyle{}{#1}$}}
\newcommand{\different}[1]{{\color{red} #1}}
% Theorems
\newtheorem{thm}{Theorem}[section]
\newtheorem{cor}[thm]{Corollary}
\newtheorem{lem}[thm]{Lemma}
\newtheorem{remark}[thm]{Remark}
\newtheorem{defn}[thm]{Definition}

% Math stuffs
\newcommand{\reach}{\ensuremath{\mathrel{\sqsubseteq}}}
\newcommand{\breach}{\ensuremath{\mathrel{\sqsupseteq}}}
\newcommand{\pto}[1]{\ensuremath{\rightharpoonup}}
\newcommand{\pow}[1]{\ensuremath{\mathcal{P}(#1)}}
\newcommand{\powfin}[1]{\ensuremath{\mathcal{P_{\mathrm{fin}}}(#1)}}
\newcommand{\card}[1]{\ensuremath{\left\vert #1 \right\vert}}
\newcommand{\real}{\ensuremath{\mathbb{R}}}
\newcommand{\nat}{\ensuremath{\mathbb{N}}}
\newcommand{\cbult}{\ensuremath{\mathrm{CBUlt}}}
\newcommand{\cle}{\ensuremath{\lesssim}}
\newcommand{\ceq}{\ensuremath{\cong}}
\newcommand{\relR}{\ensuremath{\mathrel{\mathcal{R}}}}
\newcommand{\relS}{\ensuremath{\mathrel{\mathcal{S}}}}
\newcommand{\AND}{\ensuremath{\mathrel{\wedge}}}
\newcommand{\OR}{\ensuremath{\mathrel{\vee}}}
\newcommand{\op}{\ensuremath{\mathsf{op}}}
\newcommand{\den}[1]{\ensuremath{\llbracket #1 \rrbracket}}
\newcommand{\defs}{\ensuremath{\mathrel{\triangleq}}}
\newcommand{\bifix}{\ensuremath{\mathsf{bifix}}}
\newcommand{\disjoint}{\ensuremath{\mathop{\#}}}
\newcommand{\hole}{\ensuremath{\square}}

\DeclareMathOperator{\dom}{Dom}

% Sets
\newcommand{\worlds}{\ensuremath{\mathrm{World}}}
\newcommand{\assignables}{\ensuremath{\mathrm{Assignable}}}
\newcommand{\types}{\ensuremath{\mathrm{Type}}}
\newcommand{\typesEnv}{\ensuremath{\mathrm{TypeEnv}}}
\newcommand{\term}{\ensuremath{\mathrm{Term}}}
\newcommand{\urel}{\ensuremath{\mathrm{URel}}}

% judgments
\newcommand{\guardJ}[2]{\ensuremath{\text{$#1.\,#2$\ \textsf{guarded}}}}
\newcommand{\hasE}[2]{\ensuremath{#1 \mathrel{:} #2}}
\newcommand{\hasM}[2]{\ensuremath{#1 \mathrel{\div} #2}}
\newcommand{\hasKJ}[2]{\ensuremath{#1 \vdash \hasE{#2}{\mathrm{\kind}}}}
\newcommand{\hasTJ}[4]{\ifthenelse{\isempty{#1}}%
  {\ensuremath{#2 \vdash \hasE{#3}{#4}}}%
  {\ensuremath{#1;#2 \vdash \hasE{#3}{#4}}}}
\newcommand{\hasEJ}[6]{\ifthenelse{\isempty{#1}}%
  {\ensuremath{#2; #3 \vdash \hasE{#4}{#5}}}%
  {\ensuremath{#1; #2; #3 \vdash \hasE{#4}{#5}}}}
\newcommand{\hasESigJ}[5]{\ensuremath{#1; #2 \vdash_{#3} \hasE{#4}{#5}}}
\newcommand{\hasMJ}[5]{\ensuremath{#1; #2 \vdash_{#3} \hasM{#4}{#5}}}
\newcommand{\hasCEJ}[9]{\ensuremath{#5 : (#1; #2 \vdash_{#3} #4) \rightsquigarrow (#6; #7 \vdash_{#8} #9)}}

\newcommand{\subKJ}[3]{\ifthenelse{\isempty{#1}}%
{\ensuremath{{#2} \leq {#3}}}%
{\ensuremath{{#1} \vdash {#2} \leq {#3}}}}

\newcommand{\equivESigJ}[6]{\ensuremath{#1; #2 \vdash_{#3} \hasE{#4 \cong #5}{#6}}}
\newcommand{\equivMJ}[6]{\ensuremath{#1; #2 \vdash_{#3} \hasM{#4 \cong #5}{#6}}}

\newcommand{\step}[2]{\ensuremath{#1 \mapsto #2}}
\newcommand{\steps}[2]{\ensuremath{#1 \mapsto^* #2}}
\newcommand{\stepM}[4]{\ensuremath{(#1, #2) \mapsto (#3, #4)}}
\newcommand{\stepsM}[4]{\ensuremath{(#1, #2) \mapsto^* (#3, #4)}}

\newcommand{\dec}[4][i]{\ensuremath{#2 \rhd^{#1} #3 : #4}}
\newcommand{\valueJ}[1]{\ensuremath{#1\ \mathsf{value}}}
\newcommand{\finalJ}[2]{\ensuremath{(#1, #2)\ \mathsf{final}}}

% language
\newcommand{\kind}{\ensuremath{\mathsf{kind}}}
\newcommand{\later}{\ensuremath{{\blacktriangleright}}}
\newcommand{\rec}[2]{\ensuremath{\mu #1.\, #2}}
\newcommand{\fn}[2]{\ensuremath{#1 \to #2}}
\newcommand{\tp}{\ensuremath{\mathsf{T}}}
\newcommand{\unit}{\ensuremath{\mathsf{unit}}}
\newcommand{\cmd}[1]{\ensuremath{\mathsf{cmd}(#1)}}

\newcommand{\ap}[2]{\ensuremath{#1\ #2}}
\newcommand{\lam}[3]{\ensuremath{\lambda #1 {:} #2.\, #3}}
\newcommand{\into}[1]{\ifthenelse{\isempty{#1}}%
  {\ensuremath{\mathsf{in}}}%
  {\ensuremath{\mathsf{in}\,#1}}}
\newcommand{\out}[1]{\ifthenelse{\isempty{#1}}%
  {\ensuremath{\mathsf{out}}}%
  {\ensuremath{\mathsf{out}\,#1}}}
\newcommand{\delay}[1]{\ifthenelse{\isempty{#1}}%
  {\ensuremath{\mathsf{next}}}%
  {\ensuremath{\mathsf{next}\,#1}}}
\newcommand{\letdelay}[3]{\ensuremath{\mathsf{let\ next\ } #1 = #2 \mathsf{\ in\ } #3}}
\newcommand{\all}[3]{\ensuremath{\forall #1 {:} #2.\, #3}}
\newcommand{\allNoKind}[2]{\ensuremath{\forall #1.\, #2}}

\newcommand{\Ap}[2]{\ensuremath{#1[#2]}}
\newcommand{\Lam}[3]{\ensuremath{\Lambda #1 {:} #2.\, #3}}

\newcommand{\LamNoKind}[2]{\ensuremath{\Lambda #1.\, #2}}
\newcommand{\ret}[1]{\ensuremath{\mathsf{ret}(#1)}}
\newcommand{\get}[1]{\ensuremath{\mathsf{get}[#1]}}
\newcommand{\set}[2]{\ensuremath{\mathsf{set}[#1](#2)}}
\newcommand{\dcl}[3]{\ensuremath{\mathsf{dcl}\ #1 := #2\ \mathsf{in}\ #3}}
\newcommand{\bnd}[3]{\ensuremath{\mathsf{bnd}\ #1 \gets #2;\ #3}}

\newcommand{\zap}{\ensuremath{\circledast}}


\title{The Next 700 Failed Step-Index-Free Logical Relations}
\author{Daniel Gratzer}
\date{\today}

\begin{document}
\begin{abstract}
  An important question in programming languages is the study of
  program equivalence.
  %%
  %% "the study of X" is not the "question", X is the question. Rephrase - JMS
  %%
  This is done typically with the construction of
  a relational denotational model or a syntactic analogue, called a
  logical relation. Logical relations have proven to be an effective
  tool for analyzing programs and lending formal weight to ideas like
  data abstraction and information hiding. A central difficulty with
  logical relations is their fragility; it has proven to be a
  challenge to scale logical relations to more realistic languages. In
  this thesis, we discuss a number of methods for the construction of
  a logical relation for a language with general references. None of
  these methods are sufficient to define a logical relation without
  resorting to step-indexing, demonstrating the difficulty of
  expressing the recursive structure of higher-order references.
\end{abstract}
\maketitle

In the study of programming languages a great number of important
questions can be reduced to questions about the equality of
programs. The verification of a compiler pass is nothing but a
question of equality of the naive program and its optimized version,
% na\"ive
% comma splice: please just rephrase this entire sentence
an optimized data structure may be shown to be correct relative to a
much simpler reference solution,
% comma splice
a type theory hinges on the
construction of definitional equality and so on. Given the role that
equality plays in programming languages, a great variety of
mathematical tools have been developed to analyze the various notions
of equality possible in a programming language.

When in this thesis equality is discussed, it is generally meant to be
\emph{contextual equivalence}. Contextual equivalence is formally
defined later (see
Definition~\ref{defn:language:contextualEquivalence}) but informally
contextual equivalence of two programs, $e_1 \cong e_2$, expresses
that $e_1$ and $e_2$ are internally indistinguishable. This means that
there is no program containing $e_1$ so that if $e_1$ is replaced with
$e_2$ then the program returns a different result. This notion of
equivalence is appealing because there is no aspect of a program we
care about other than how it computes.
% Better to say "what" it computes, since how it computes can change, and we don't want to care about that.
Contextual equivalence allow to
% "allow to" ==> "allows one/us to"
ignore unimportant differences in the precise way that the answer of a
program is computed and focus instead on the answer itself. This means
% delete "well-behaved" :
that two well-behaved implementations of a data structure, for
% replace "are" with "can be"; "if" with "when"
instance, are contextually equivalent even if one is far more complex
and efficient than the other.


For all the appeal of contextual equivalence, it is very difficult to
establish that two programs are contextually equivalent. In order to
establish that $e_1 \cong e_2$ it is necessary to quantify over all possible
programs using $e_1$ and $e_2$ and reason about their behavior. This
includes programs which do not really make use of $e_1$ or $e_2$ such
as $\ap{(\lam{x}{\fn{\unit}{\tau}}{1})}{\lam{x}{\unit}{e_1}}$. This
lack of constraints on how $e_1$ or $e_2$ is used is precisely what
makes contextual equivalence so useful but it makes even the most
basic proofs involved affairs.
% What follows is highly a-historical. The craze over contextual equivalence
% blah-blah is relatively recent in comparison to techniques like denotational
% semantics for characterizing the equivalence of programs, and even logical
% relations (which were not invented to deal with contextual equivalence, but
% came from general mathematical considerations).
In order to compensate for this, a
variety of tools have been developed to simply the process of
establishing when $e_1 \cong e_2$.

% replace "these" with "such"
Broadly speaking, there are two classes of these tools. One may
consider denotational approaches, where in the equality of programs is
reduced to the question of the equality of normal mathematical
objects. This line of study begins with Dana Scott's investigations of
the lambda calculus~\citep{TODO-CITE-SCOTT}. These tools have been
immensely effective when they may be applied but they are difficult to
use. Complex programming languages often make use of extremely
sophisticated mathematical objects and using the model requires
understanding them. This has meant that denotational semantics is
traditionally out of reach for the verification of programs by an
average programmer or anyone besides a domain expert.

On the other hand, there are syntactic approaches to equality. Some of
these date back to the original study of the lambda calculus and its
reduction properties. Syntactic tools tend to be simpler to use,
relying only one elementary mathematics and an understanding of the
syntax itself. For equality, the tool of choice when working
syntactically is a logical relation~\citep{TODO-TAIT-AMAL}. Dating
back to~\citet{TODO-TAIT} logical relations crucially define a
type-indexed notion of equality. Logical relations have proven
important for validating certain obviously desirable properties such
as function extensionality.
% isn't it much the other way around? Parametric reasoning is surely
% not inspired by logical relations? Also, I feel there is no need
% to drop the reference to the "purely abstract" (not sure what this means)
% denotational stuff. Anyway, the first attempts to model parametricity
% were denotational anyway, were they not? Further, it is not the importance
% of parametricity that encouraged fresh attempts at a denotational story, but
% rather the unsatisfactory nature of the existing models of parametricity.
They have also given rise to a powerful
theory of data abstraction called
parametricity~\citep{TODO-REYNOLDS}. Parametric reasoning as inspired
by logical relations has been so important that there have even been
attempts to reconstruct it in a purely abstract denotational
sense~\citep{TODO-FIBRATIONAL}.

% "in for us"
% I don't understand what is meant by the double subscript here.
% I assume it is a typo?
To be precise, a logical relation in for us is a family of sets
indexed by the types of the language, $(R_\tau)_\tau$. It is
constructed by induction over the types indexing it so that
$R_{\fn{\tau_1}{\tau_2}}$ is defined in terms of $R_{\tau_1}$ and
$R_{\tau_2}$. For each $\tau$, the following property is expected to
hold:
\[
  (e_1, e_2) \in R_\tau \implies
  \hasE{\cdot}{e_1, e_2}{\tau} \land e_1 \cong e_2
\]
This property expresses the soundness of the logical relation with
% regards to => respect to
regards to contextual equivalence: logical equivalence implies
contextual equivalence. The reverse property, completeness, is
desirable but often unachievable\footnote{Instead a variety of
  techniques for \emph{forcing} completeness to hold are
  employed. These amount to adding all the missing identifications to
  the logical relation. While theoretically desirable, this is useless
  for the problem of actually establishing an
  equivalence.}. Completeness is usually unnecessary for establishing
certain concrete equivalences so in this context it is less
important. Finally, an important property of logical relations which
is difficult to capture formally is that it is much easier to prove
that $(e_1, e_2) \in R_\tau$ than to directly show that
$e_1 \cong e_2$. Typically, $R_\tau$ is meant to capture the logical
action of $\tau$; it expresses the precise set of observations
possible to make of expressions of type $\tau$ without all the
duplication of contextual equivalence. For instance, in order to show
that $(e_1, e_2) \in R_{\fn{\tau_1}{\tau_2}}$ it suffices to show the
following.
\[
  \forall (a_1, a_2) \in R_{\tau_1}.
  \ (\ap{e_1}{a_1}, \ap{e_2}{a_2}) \in R_{\tau_2}
\]
% having a logical relation ==> having a notion of logical equivalence;
% and the two halves of this sentence feel kind of mutually redundant,
% so maybe there is a way to rephrase it.
Given the obvious appeal of having a logical relation defined for a
language, it would seem that the construction of a logical relation
characterizing equality is a natural first step in the study of a new
programming language. The central stumbling block to this goal is that
logical relations are difficult to extend to new programming languages
and especially to new types and computational effects.

% "For an instance of where" ==> "To see where"
For an instance of where trouble might arise, consider a language with
polymorphic types: $\allNoKind{\alpha}{\tau}$. What should the
% ==> Phrased differently
definition of $R_{\allNoKind{\alpha}{\tau}}$ be? Phrased different,
the question is when are two polymorphic functions
% put a colon after "following"
indistinguishable. One might expect a definition like the following.
\[
  (e_1, e_2) \in R_{\allNoKind{\alpha}{\tau}} \defs
  \forall \tau'.\ (\Ap{e_1}{\tau'}, \Ap{e_2}{\tau'}) \in R_{[\tau'/\alpha]\tau}
\]
This does correctly characterize contextual equivalence but it is
ill-suited as a definition in a logical relation because it will not
be well-founded. A logical relation is constructed by induction on the
% might be clearer to say "...on the syntactic types"; add comma after "types"
types and with impredicative polymorphism there is absolutely no
reason why $\tau'$ should be smaller than
% comma splice, please fix ! good l0rd
$\allNoKind{\alpha}{\tau}$. This is not a minor issue, defining a
correct logical relation for a language with impredicative
polymorphism requires the novel method of
candidates~\citep{TODO-GIRARD-TAIT-REYNOLDS-TAIT-TAIT}. This problem has been a recurring
issue: features in programming languages tend to have some recursive
structure which prevents them from easily fitting into the inductive
% You need a comma after "features" for this to make sense
definition of a logical relation. For many features clever
constructions have been found which circumvent well-foundedness
issues: for parametric polymorphism~\citep{TODO-GIRARD}, for
first-order state~\citep{TODO-PITTS-AND-STARK}, simple
exceptions~\citep{TODO-I-DUNNO-SURELY-NOT-NUPRL}, and for recursive
% you need a comma after "work" for this to make sense
types~\citep{TODO-CRARY-AND-HARPER}. Despite this work logical
relations are still a long way away from being able to cope with a
full programming language. The state of the art for dealing with
features like higher-order references, nontrivial control passing, or
concurrency is to use step-indexing.

% the following paragraph is way too long; just add a few paragraph breaks
% at good moments.
Step-indexing is a technique first proposed by \citet{TODO-APPEL}. It
% "if you the" :
is based on a simple but ingenious idea: if you the logical relation
is not well-defined by induction on the type just add a number which
% please never use "induct" as a verb; sometimes people use this in
% speech, but it is not allowed in scholarly writing.
decreases and induct on that. This idea means that a logical relation
is no longer a type-indexed relation but a relation indexed by a
number (called a \emph{step}) and a type, $(R_\tau^i)_{i, \tau}$. The
meaning of a relation at step $i$ and type $\tau$ is that two programs
are related if they are indistinguishable at type $\tau$ for $i$
steps. Two programs are related for $i$ steps intuitively if it will
take at least $i$ steps to tell them apart. For instance, if a program
$e_1$ runs to $\tt$ in 10 steps while $e_2$ runs to false in 20, it
takes 20 steps to distinguish them since we must wait for $e_2$ to
finish running before they can be compared. This idea of steps has
proven to be incredibly robust and easily extend to many different
settings~\citep{TODO-SUCCESSES-OF-STEP-INDEXING}. There is a price to
be payed for this extra flexibility. Firstly, steps will now pervade
the definition of the logical relation and any proof now must contain
pointless bookkeeping activity in order to keep track of
them. Secondly, the evaluation behavior of the programs under
consideration has begun to matter again. The goal of contextual
equivalence was to erase it from consideration but now there are
instances where in order to use a logical relation a program must be
% may be clearer to say "redundant operations" rather than "NO-OP statements"
modified with spurious NO-OP statements in order to make it take extra
steps. These issues are even present when just using the logical
% validate => establish
relation to validate contextual
equivalences~\citep{TODO-TRANSFINITE}. In practice, this has meant
% you need a comma after "used" for this to make sense
that while step-indexing is incredibly widely used it is almost
universally disliked\footnote{Citation: I walked around at POPL and
  the 5 people I asked seemed like REALLY bummed about it.}.

The structure of this thesis as follows. In Section~\ref{sec:language}
an ML-like language is described that will serve as target for the
logical relation. In Section~\ref{sec:step-indexing} a step-indexed
logical relation for this language is constructed and discussed,
especially to highlight its short-comings. In
Section~\ref{sec:domains} the first failed step-index-free logical
relation is described, the most naive approach centered around using
domain theory. In Section~\ref{sec:handedness} a variety a number of
step-index-free logical relations are discussed for languages powerful
enough to encode general references, largely built around (guarded)
recursive kinds.

%%% Local Variables:
%%% mode: latex
%%% TeX-master: "../main"
%%% End:

\section{An ML-like Language with General References}\label{sec:language}

In order to make the discussion of logical relations more concrete, a
particular language is necessary. In this section we develop a core
calculus suitable for studying the effects of general references on
reasoning.

The language under consideration here is heavily influenced by the
Modernized Algol discussed by Harper~\citep{TODO:PFPL}. It features a
syntactic separation between commands and expressions. Expressions are
characterized by being unable to depend on in any way on the
heap. Commands may modify the heap using assignables, a mutable
``variable'' that makes clear the binding structure (nominal rather
than substitutive). A crucial component of the system is a modality
for internalizing and suspending commands to treat them as
expressions. Arguably this language is heavier-weight than the ML-like
languages usually discussed which just allow references in any
location. It does confer that advantage that handling state is now a
question of defining the logical action of the command modality and
its complexity is not littered throughout the full logical
relation. Indeed, the definitions of the logical relation at $\to$ or
$\forall$ are almost completely unchanged.

The syntax of our language has three sorts: commands, expressions, and
types defined by the following grammar.
\[
  \begin{array}{lcl}
    \tau & ::= & \alpha \mid \fn{\tau}{\tau} \mid \allNoKind{\alpha}{\tau}
    \mid \cmd{\tau}\\
    c & ::= & \ret{e} \mid \get{\alpha} \mid \set{\alpha}{e} \\
      & \mid & \dcl{\alpha}{e}{c} \mid \bnd{x}{e}{c}\\
    e & ::= & x \mid \ap{e}{e} \mid \lam{x}{\tau}{e} \mid
              \LamNoKind{\alpha}{e} \mid \cmd{e}
  \end{array}
\]
Only one point of this syntax must be clarified, which is the
unfortunate coincidence of $\alpha$s. In an attempt to maintain
consistency with the standard literature on System F and
Harper~\citep{TODO:PFPL} $\alpha$ here refers either to an assignable
(a symbol) or a type variable. Crucially, assignables are \emph{not}
variables. Assignables are bound by the operators of our language and
they do $\alpha$-vary as a variable might but they may not be
substituted for. This justifies comparing bound symbols for equality;
they are placeholders and contain an intrinsic notion of identity in
the form of a binding site. It is nonsensical to talk about a rule
like the following.
\[
  \inferrule{ }{\step{\dcl{\alpha}{v}{c}}{[v/\alpha]c}}
\]
In some sense, this is similar to the confusion that many programmers
have when discussing languages with variables: C-like languages do not
permit substitution because they do not possess variables but rather
assignables. In C a rule like the above is clearly false and leads to
statements such as \verb+1 = 2+. One can understand the difference
between $\bnd{x}{e}{c}$ and $\dcl{x}{e}{c}$ as the difference between
\emph{variables}, defined through substitution, and \emph{assignables}
which are defined through binding identity. The former is like a
let-binding while the latter is closer to a declaration in C. In our
language this separation exists which is why the operators for reading
and writing a mutable cell are indexed by symbol rather than taking an
arbitrary term. There is much to be said on the subject of symbols
and, more generally, nominal binding but it is sadly out-of-scope for
this discussion. The interested reader is referred
to~\citep{TODO-NOMINAL-STUFF}.

The static semantics of the language are divided into three judgments,
the first of which is the judgment ensuring a type is a
well-formed. Informally, a type is well formed in a context $\Delta$
if $\Delta$ contains all the free variables of the type.
\begin{mathpar}
  \declareJudgement{\hasTJ{}{\Delta}{\tau}{\tp}}\\
  \inferrule{
    \alpha \in \Delta
  }{\hasTJ{}{\Delta}{\alpha}{\tp}}\and
  \inferrule{
    \hasTJ{}{\Delta}{\tau_1}{\tp}\\
    \hasTJ{}{\Delta}{\tau_2}{\tp}
  }{\hasTJ{}{\Delta}{\fn{\tau_1}{\tau_2}}{\tp}}\and
  \inferrule{
    \hasTJ{}{\Delta, \alpha}{\tau}{\tp}
  }{\hasTJ{}{\Delta}{\allNoKind{\alpha}{\tau}}{\tp}}\and
  \inferrule{
    \hasTJ{}{\Delta}{\tau}{\tp}
  }{\hasTJ{}{\Delta}{\cmd{\tau}}{\tp}}
\end{mathpar}
In order to explain the statics of expressions and commands two
judgments are necessary and they must depend on each other. This
mutual dependence is a result of the $\cmd{-}$ modality which
internalizes the command judgment. First the expression judgment is
given, it is completely standard except that it must also be fibered
over a specification of the available assignables. This extra context
is necessary in order to make sense of the binding done in
$\mathsf{dcl}$.
\begin{mathpar}
  \declareJudgement{\hasESigJ{\Delta}{\Gamma}{\Sigma}{e}{\tau}}\\
  \inferrule{
    x : \tau \in \Gamma
  }{\hasESigJ{\Delta}{\Gamma}{\Sigma}{x}{\tau}}\and
  \inferrule{
    \hasESigJ{\Delta}{\Gamma, x : \tau_1}{\Sigma}{e}{\tau_2}
  }{\hasESigJ{\Delta}{\Gamma}{\Sigma}{\lam{x}{\tau_1}{e}}{\fn{\tau_1}{\tau_2}}}\and
  \inferrule{
    \hasESigJ{\Delta}{\Gamma}{\Sigma}{e_1}{\fn{\tau_1}{\tau_2}}\\
    \hasESigJ{\Delta}{\Gamma}{\Sigma}{e_2}{\tau_1}\\
  }{\hasESigJ{\Delta}{\Gamma}{\Sigma}{\ap{e_1}{e_2}}{\tau_2}}\and
  \inferrule{
    \hasESigJ{\Delta, \alpha}{\Gamma}{\Sigma}{e}{\tau}
  }{\hasESigJ{\Delta}{\Gamma}{\Sigma}{\LamNoKind{\alpha}{e}}{\allNoKind{\alpha}{\tau}}}\and
  \inferrule{
    \hasESigJ{\Delta}{\Gamma}{\Sigma}{e}{\allNoKind{\alpha}{\tau_1}}\\
    \hasTJ{}{\Delta}{\tau_2}{\tp}
  }{\hasESigJ{\Delta}{\Gamma}{\Sigma}{\Ap{e}{\tau_2}}{[\tau_2/\alpha]\tau_1}}\and
  \inferrule{
    \hasMJ{\Delta}{\Gamma}{\Sigma}{m}{\tau}
  }{\hasESigJ{\Delta}{\Gamma}{\Sigma}{e}{\tau}}
\end{mathpar}
These are the rules of System F with the exception of the final
one. This makes use of the judgment for commands, where
$\hasM{m}{\tau}$ signifies that $m$ is a command which (when executed
on the appropriate heap) will run to a $\ret{v}$ for some
$\hasE{v}{\tau}$. It is worth noting that $\Sigma$ in this judgment is
unimportant except in this rule as well. It is only necessary to track
the available assignables because without this information it is not
possible to determine if a command is well-typed. The rules for
commands are as follows.
\begin{mathpar}
  \inferrule{
    \hasESigJ{\Delta}{\Gamma}{\Sigma}{e}{\tau}
  }{\hasMJ{\Delta}{\Gamma}{\Sigma}{\ret{e}}{\tau}}\and
  \inferrule{
    \alpha \div \tau \in \Sigma
  }{\hasMJ{\Delta}{\Gamma}{\Sigma}{\get{\alpha}}{\tau}}\and
  \inferrule{
    \alpha \div \tau \in \Sigma\\
    \hasESigJ{\Delta}{\Gamma}{\Sigma}{e}{\tau}
  }{\hasMJ{\Delta}{\Gamma}{\Sigma}{\set{\alpha}{e}}{\tau}}\and
  \inferrule{
    \hasESigJ{\Delta}{\Gamma}{\Sigma}{e}{\tau_2}\\
    \hasMJ{\Delta}{\Gamma}{\Sigma, \alpha : \tau_2}{m}{\tau_1}
  }{\hasMJ{\Delta}{\Gamma}{\Sigma}{\dcl{\alpha}{e}{m}}{\tau_1}}\and
  \inferrule{
    \hasESigJ{\Delta}{\Gamma}{\Sigma}{e}{\cmd{\tau_1}}\\
    \hasMJ{\Delta}{\Gamma, x : \tau_1}{\Sigma}{m}{\tau_2}
  }{\hasMJ{\Delta}{\Gamma}{\Sigma}{\bnd{x}{e}{m}}{\tau_2}}\and
\end{mathpar}

%%% Local Variables:
%%% mode: latex
%%% TeX-master: "../main"
%%% End:

\section{A Step-Indexed Logical Relation}

Before diving into the various approaches for constructing a logical
relation without step-indexing, it is well worth the time to see how
a logical relation can be done with it. The purpose of this section is
to sketch the complication intrinsic to any logical relation and show
how step-indexing obliterates them, though at a high cost.

Our logical to begin with a mapping from types to semantic types
(merely sets of terms). In order to handle impredicative polymorphism
Girard's method~\citep{Girard:71,Girard:72}, see \citet{TODO-PFPL} for
a comprehensive explanation of the technique. This means that our
logical relation is of the form
\[
  \den{-}_{-} : \types \to \typesEnv \to \pow{\term \times \term}
\]

The central challenge is of course the meaning of $\den{\cmd{\tau}}$:
the action of the logical relation at commands. At an intuitive level,
for two commands are rather like (partial) functions: they map heaps to heaps
and a return value. Drawing inspiration from how logical relations for
functions are defined, we might write the following for the definition
the logical relation.
\begin{align*}
  \den{\cmd{\tau}}_\eta& \triangleq \{(e_1, e_2) \mid\\
  &\exists m_1, m_2.\ \steps{e_i}{\cmd{m_i}} \land{}\\
  &\forall h_1 \sim h_2.
  \ (m_1, h_1) \Downarrow \iff (m_2, h_2) \Downarrow \land{}\\
  &\quad \forall v_1, h_1', v_2, h_2'.
  \ (\stepsM{m_1}{h_1}{\ret{v_1}}{h_1'} \land \stepsM{m_2}{h_2}{\ret{v_2}}{h_2'})\\
  &\qquad \implies (h_1' \sim h_2' \land (v_1, v_2) \in \den{\tau}_\eta)
\end{align*}
Here left undefined is the definition of $\sim$ between two
heaps. This is in fact a major issue because there appears to be no
good way to identify when two heaps ought to be equal. The first issue
here is that semantic equality of terms (be it contextual or logical)
is type-indexed. This means that in order to compare heaps pointwise
for equality (a reasonable though still wrong idea) requires that we
at least know the types of the entries. Furthermore, we shouldn't
compare these heaps for equality at all locations necessarily, two
heaps should only need to agree on the cells that the programs are
going to use. This is a significant concept if we want to prove
programs to be equivalent which do not use the heap identically. For
instance, consider the two programs:
\[
  \dcl{\alpha}{1}{\ret{\cmd{\get{\alpha}}}} \qquad\qquad
  \ret{\cmd{\ret{1}}}
\]
These are contextually equivalent (the assignable of the first program
is hidden from external manipulation) and yet they allocate in
different ways. So $\sim$ must not be \emph{merely} pointwise equality
in the most general case. Additionally, proving that these two
programs are equal requires showing that $h_1 ~ h_2$ if and only if
$h_1(\alpha) = 1$. That is, this program doesn't merely require that
heap cells contain values of some syntactic type, but they may need to
belong to an arbitrary semantic type. In order to reconcile these
constraints, one thing is clear: the logical relation must somehow
vary depending on the state that the heap is supposed to be in. It is
simply not the case that programs that are equivalent in a heap where
no cells are required to exist if and only if they're equivalent in a
heap where one cell is required to exist.

%%% Local Variables:
%%% mode: latex
%%% TeX-master: "../main"
%%% End:

\section{Tying the Knot Using Domains}

\section{Understanding State through Guarded Recursion}

\subsection{A Logical Relation in Ultrametric Spaces}

%%% Local Variables:
%%% mode: latex
%%% TeX-master: "../main"
%%% End:

\section{Conclusions}


\bibliographystyle{plainnat}
\bibliography{citations}
\end{document}
